\documentclass[a4paper, fontsize=10bp]{article} % преамбула документа
\usepackage{scrextend}
\usepackage[left=2.0cm, right=1.5cm, top=2.0cm,
bottom=2.0cm]{geometry}
\usepackage[utf8]{inputenc}
\usepackage[T2A]{fontenc}
\usepackage{hyperref}
\usepackage[english, russian]{babel}
\usepackage{amsmath,amsfonts,amssymb}
\usepackage{graphicx}
\usepackage{setspace}
\usepackage{fontsize}

\title{Переписанные на \LaTeX вопросы для экзамена по информатике}
\author{Морозов Н.Ю}
\date{\today}


\begin{document} % тело документа
\selectlanguage{russian}

\maketitle

\newpage

\tableofcontents

\newpage

\singlespacing

\section{Базовые конструкции программирования: синтаксис и семантика языков 
высокого уровня; переменные, типы, выражения и присваивания; простейший ввод/вывод;} 
\ 

{\bfseries Языки программирования} – это формальные искусственные языки. Как и естественные
языки, они имеют алфавит, словарный запас, грамматику и синтаксис, а также семантику.

{\bfseries Алфавит} – разрешенный к использованию набор символов, с помощью которого могут 
быть образованы слова и величины данного языка.

{\bfseries Синтаксис} – система правил, определяющих допустимые конструкции языка 
программирования из букв алфавита.

{\bfseries Семантика} – система правил однозначного толкования каждой языковой конструкции, 
позволяющих производить процесс обработки данных.

Взаимодействие синтаксических и семантических правил определяет основные понятия языка, такие 
как операторы, идентификаторы, константы, переменные, функции, процедуры и т.д. В отличие от 
естественных, язык программирования имеет ограниченный запас слов (операторов) и строгие 
правила их написания, а правила грамматики и семантики, как и для любого формального языка, 
явно однозначно и четко сформулированы.

Языки программирования, ориентированные на команды процессора и учитывающие его особенности, 
называют языками низкого уровня. «Низкий уровень» не означает неразвитый, имеется в виду, что 
операторы этого языка близки к машинному коду и ориентированы на конкретные команды процессора.

Языком самого низкого уровня является ассемблер. Программа, написанная на нем, представляет 
последовательность команд машинных кодов, но записанных с помощью символьных мнемоник. В таком 
случае программист получает доступ ко всем возможностям процессора. Однако, при этом требуется 
хорошо понимать устройство компьютера, а использование такой программы на компьютере с процессором 
другого типа невозможно. Такие языки программирования используются для написания небольших системных 
приложений, драйверов устройств, модулей стыковки с нестандартным оборудованием.

Языки программирования, имитирующие естественные, обладающие укрупненными командами, ориентированные 
«на человека», называют языками высокого уровня. Чем выше уровень языка, тем ближе структуры данных 
и конструкции, использующиеся в программе, к понятиям исходной задачи. Особенности конкретных 
компьютерных архитектур в них не учитываются, поэтому исходные тексты программ легко переносимы на 
другие платформы, имеющие трансляторы этого языка. Разрабатывать программы на языках высокого уровня 
с помощью понятных и мощных команд значительно проще, число ошибок, допускаемых в процессе 
программирования, намного меньше. В настоящее время насчитывается несколько сотен таких языков 
(без учета их диалектов).

Программы оперируют с различными данными, которые могут быть простыми и структурированными. Простые 
данные - это целые и вещественные числа, символы и указатели (адреса объектов в памяти). 
Структурированные данные - это массивы и структуры;

{\bfseries Переменные} - это именованные области памяти, используемые для хранения данных. Каждая 
переменная имеет свой тип данных, который определяет, какой вид информации может быть сохранен в данной
переменной.

В языке различают понятия "тип данных"\  и "модификатор типа". {\bfseries Тип данных} - это, например, 
целый, а модификатор - со знаком или без знака. Целое со знаком будет иметь как положительные, так и 
отрицательные значения, а целое без знака - только положительные значения. В языке Си можно выделить пять 
базовых типов, которые задаются следующими ключевыми словами:

\begin{itemize}
    \item char - символьный; 
    \item int - целый;
    \item float - вещественный;
    \item double - вещественный двойной точности;
    \item void - не имеющий значения;
\end{itemize}
Дадим им краткую характеристику:
\begin{quotation}{Дадим им краткую характеристику:}
    \begin{enumerate}
        \item Переменная типа char имеет размер 1 байт, её значениями являются символы.
        \item Размер переменной типа int в стандарте языка Си не определён. В болшинстве систем 
                программирования размер переменной типа int соотвествует размеру целого машинного 
                слова. В этом случае знаковые значения этой переменной могут лежать в диапазоне от 
                $-32768$ до $32767$.
        \item Ключевое слово float позвляет определить переменные вещесвенного типа. Их значения имеют 
                дробную часть, отделяемую точкой, например: $-5.6, 31.28$ и т.п. Вещественные числа 
                могут быть записаны также в форме с плавающей точкой, например: $-1.09e+4$. Число перед 
                символом "e" называется мантиссой, а после "e" - порядком. Она может принимать значения 
                в диапазоне от $3.4e-38$ до $1.7e+308$.
        \item Ключевое слово double позвляет определить вещественную переменную двойной точности.
                Она занимет в памяти в два раза больше места, чем переменная типа float. Переменная 
                типа double может принимать значения в диапазоне от $1.7e-308$ до $1.7e+308$.
        \item Ключевое слово void (не имеющий значения) используется для нейтрализации значения объектов, 
                например, для объявления функции, не вовзращающей никаких значений.
    \end{enumerate}
\end{quotation}
Объект некоторого базового типа может быть модифицирован. С этой целью используются специальные ключевые 
слова, называемые модификаторами. В стандарте ANSI языка Си имеются следующие модификаторы типа:
\begin{itemize}
    \item unsigned
    \item signed
    \item short
    \item long
\end{itemize}   

\noindent Модификаторы записываются перед спецификаторами типа, например: unsigned char. Если после 
модификатора опущен спецификатор, то компилятор предполагает, что этим спецификатором является int.
Таким образом следующие строки:

\qquad \: long a;

\qquad \: long int a;

\noindent Все переменные до их использования должны быть определены (объявлены). При этом задается 
тип, а затем идет список из одной или более переменных этого типа, разделенных запятыми. Например:
\\
int a, b, c; 
\\
char x, y; \bigskip 
\\
Переменные в языке Си могут быть инициализированы при их определении:

\begin{quotation}
    int a = 25, h = 6;

    char g = 'Q', k = 'm';
    
    float r = 1.89;
    
    long double n = r*123;
\end{quotation}

\noindent В языке возможны глобальные и локальные объекты. Первые определяются вне функций и, следовательно,
доступны для любой из них. Локальные объекты по отношению к функциям являются внутренними. Они начинают 
существовать, при входе в функцию и уничтожаются после выходы из неё. 

\ 

int a; \qquad \qquad \:/* Объявление глобальной переменной */

\ 

int function(int b, char c); \: /* Объявление функции (т.е. описание её заголовка) */

\ 

void main(void)

\ 

\{ \qquad \qquad \qquad //Тело программы

\qquad int d, e; \quad  //Определение локальных переменных

\qquad float f;  \quad \: //Определение локальной перемной

\qquad ...

\}

\ 

\noindent {\bfseries Операции ввода/вывода} в языке Си организованы посредством библиотечных функций.

\noindent Самый простой механизм ввода - чтение по одному символу из стандартного входного потока с помощью 
функции getchar():

x = getchar(); \\
присваевает переменной x очередной вводимый символ. Переменная x должна иметь символьный или целый тип. \\
Другая функция - putchar(x) выдаёт значение переменной x в стандартный выходной поток:

int putchar(int);

Объявления getchar() и putchar() сделаны в заголовочном файле stdio.h содержащем описания заголовков библиотечных
функций стандартного ввода/вывода. Подключение осуществляется с помощью директивы препроцессора:

\# include <stdio.h>\\
Функция printf() обеспечивает форматированный вывод. Ее можно записать в следующем формальном виде:

printf( "управляющая срока"\ , аргумент\_1, аргумент\_2);

\noindent Управляющая строка содержит компоненты трех типов: обычные символы, которые просто копируются в
стартный выходной поток (выводятся на экран дисплея); спецификации преобразования, каждая из
которых вызывает вывод на экран очередного аргумента из последующего списка; управляющие 
символьные константы.

\noindent Каждая спецификация преобразования начинается со знака \% и заканчивается нектороым символом,
задающим преобразование. Между знаком \% и символом преобразования могут встречаться другие знаки в
соотвествии со следующим форматом:\\
На место символа преобразования могут быть записаны:

\begin{quotation}
    
    \noindent {\textbf c} -- значением аргумента является символ;
\\
    {\textbf d} или {\textbf i} -- значением аргумента является десятичное число;
\\
    {\textbf e} -- значением аргумента является вещественное десятичное число в экспоненциальной форме вида
    $1.23e+2$;
\\
    {\textbf f} -- значением аргумента является вещественное десятичное число с плавающей точкой;
\\
    {\textbf s} -- значением аргумента является строка символов (символы строки выводятся до тех пор, пока не 
                    встретится символ конца строки или же не будет, выведенно число символов, заданное точностью);
\\
    {\textbf u} -- значением аргумента является беззнаковое целое число;
\\
    {\textbf p} -- значением аргумента является указатель;
\\
    {\textbf n} -- применятся в операциях форматирования. Аргумент, соответсвтующий этому символу
                    спецификации, должен быть указателем на целое. В него возвращается номер позиции строки
                    (отображаемой на зкране), в которой записана спецификация \%n.

\end{quotation}

Функция printf() использует управляющую строку, чтобы определить, сколько всего аргументов и каковы
их типы. Аргументами могут быть переменные, константы, выражения, вызовы функций; главное, чтобы их
значения соотвествовали заданной спецификации.

\noindent Функция scanf() обеспечивает форматированный ввод. Ее можно записать в следующим виде:

scanf("управляющая строка"\ , аргумент\_1, аргумент\_2, ...);

\noindent Аргументы scanf() должны быть указателями на соответствующие значения. Для этого перед именем
переменной записывается символ \& -- оперция получения адреса в памяти.

\noindent Управляющая строка содержит спецификации преобразования и используется для установдения количества
и типов аргументов


\section{Условные предложения и итеративные конструкции}

\subsection*{Операторы цикла}

\indent Циклы организуются, чтобы выполнить некоторый оператор или группу



\end{document}