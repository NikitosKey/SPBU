\documentclass[a4paper, 10pt]{article}
\documentclass[a4paper, fontsize=10bp]{article} % преамбула документа
\usepackage{scrextend}
\usepackage[left=2.0cm, right=1.5cm, top=2.0cm,
bottom=2.0cm]{geometry}
\usepackage[utf8]{inputenc}
\usepackage[T2A]{fontenc}
\usepackage{hyperref}
\usepackage[english, russian]{babel}
\usepackage{amsmath,amsfonts,amssymb}
\usepackage{graphicx}
\usepackage{setspace}
\usepackage{fontsize}

\begin{document}
\noindent {\bfseries Морозов}

Вариант 6

\begin{enumerate}
    \item Написать функцию, принимающую отсортированный динамический массив целых чисел,
    его длину и значение искомого элемента и возвращающую индекс последнего элемента 
    в массиве, имеющего искомое значение, методом, аналогичным двоичному поиску (просто
    найти двоичным поиском какой-то элемент, равный искомому, а затем циклом бежать от 
    него до первого, имеющего то же значение, не годится, потому что таких элементов может 
    быть слишком много). Если в массиве такого элемента нет, нужно возвратить  «-1».
    \item Ввести с консоли несколько строк. Количество строк заранее неизвестно (указатель
    на массив строк, поставить ограничение длины массива), признак конца ввода – EOF (end
    of file – строка “end” – можно определить директивой define). Распечатать введенные
    строки, сконкатенированные по 2 – первая с последней, вторая – с предпоследней и т.д.
    Использовать динамическое выделение памяти, можно использовать функции strlen, strcpy,
    strcat.

    (int strlen(char *s) – возвращает длину строки без учета завершающего 0-символа.

    char *strcpy(char *$s1$, char *$s2$) – копирует $s2$ в $s1$, возвращает $s1$

    char *strcat(char *s1, char *s2) – добавляет $s2$ к $s1$ в конец., последним символом добавляет
    $\backslash0$, возвращает $s1$.)
    \item Написать функцию, принимающую вещественное число х и два указателя $f$ и $g$ на функини 
    типа double(double), и возвращающую результат применения $f$ к сумме $х$ и результата 
    применения $g$ к квадрату $x$
    
\end{enumerate}


\end{document}