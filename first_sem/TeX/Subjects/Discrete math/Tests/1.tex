\input ../../my_simple_preamble.tex

\begin{titlepage}
    \title{Контрольная по дискрентной математике}
    \author{Морозов Н.Ю}
    \date{}
\end{titlepage}

\begin{document}

    \maketitle  
        
    \section{}

    $ \sphericalangle \quad f : A \longrightarrow B,\ |A| = k,\ |B| = n$ при $ n \leqslant  k $ только в этом случае возможна иньекция. 

    \noindent Пусть N - искомое количество возможных отображений из $A$ в $B$, тогда

    \begin{enumerate}
        \item $N = A^k_n$, что справедливо для любых $n, k \in \mathbb{N}, \{0\} $ и удовлетворяющих условию.
        \item Также это будет верно и для тех случаев когда оба множества пустые и когда $A$ пустое, а $B$ - любое
        конечное, поскольку эти случаи не противоречат здравому смыслу и соотвестуют условиям выше.
        \item А для $ \quad f : A \longrightarrow B, \ A - \varnothing, B$ -- любое, но не пустое не выполняется неравенство 
        $ n \leqslant  k \Longrightarrow f$ не иньекция $\Longrightarrow$ этот случай не подходит.
    \end{enumerate}

    \section{}

    Mорозов $ \Longrightarrow $ N = 7

   \noindent $E_i$ -- матождиание для события вытащенно $i$ тузов.

   

    \begin{align}
        &E = \sum_{i = 0}^{4} \left(\frac{C^{7-i}_{32} \cdot C^{i}_{4}}{C^7_{36}} \cdot i \right)  =
        \sum_{i=0}^{4} \left( \frac{\frac{32!}{(7-i)!(32-7+i)!} \cdot \frac{4!}{i!(4-i)!}}{\frac{36!}{7!(36 - 7)!}} \cdot i\right)  = \frac{7}{9}\\
        &D = \sum_{i=0}^{4} \left( \frac{\frac{32!}{(7-i)!(32-7+i)!} \cdot \frac{4!}{i!(4-i)!}}{\frac{36!}{7!(36 - 7)!}} \cdot i^2\right) - \left(\frac{7}{9}\right)^2 = \frac{232}{405}
    \end{align}

    \section{}

    \begin{enumerate}
        \item $34816572 = 34816725$
        \item $34816725 = 34816752$
        \item $34816752 = 34817256$
        \item $34817256 = 34817265$
        \item $34817265 = 34817526$
        \item $34817526 = 34817562$
        \item $34817562 = 34817625$
        \item $34817625 = 34817652$
    \end{enumerate}

    \newpage

    \section{}

    \begin{figure}[h]
        \centering
        \includegraphics[width=0.8\linewidth]{кал лютый.png}

        \end{figure}

        
        
    

\end{document}