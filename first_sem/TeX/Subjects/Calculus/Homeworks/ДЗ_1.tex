\documentclass[a4paper, fontsize=12pt]{article} % преамбула документа
\usepackage{scrextend}
\usepackage[left=2.0cm, right=1.5cm, top=2.0cm,
bottom=2.0cm]{geometry}
\usepackage[utf8]{inputenc}
\usepackage[T2A]{fontenc}
\usepackage{hyperref}
\usepackage[english, russian]{babel}
\usepackage{amsmath,amsfonts,amssymb}
\usepackage{graphicx}
\usepackage{setspace}
\usepackage{fontsize}


\begin{titlepage}
    \title{Homework 1}
    \date{}
\end{titlepage}


\begin{document}
    \maketitle

    \section{}

    умею

    \section{\normalsize \normalfont }

    1. Определить значение истинности высказывания, считая, что все переменные
    пробегают множество действительных чисел.

    $$\exists a : \forall b \  \exists x : x^2 + ax + b = 0$$ \smallskip

    Решение:
    %$x^2 + 2345678654567x + 10000000000000000000000000000000000000000 = 0  $
    $$x = - \frac{a}{2} - \frac{\sqrt{a^{2} - 4 b}}{2}, \  x = - \frac{a}{2} + \frac{\sqrt{a^{2} - 4 b}}{2}$$\smallskip
    При $a^2 < 4b$ данное выражение ложно, так как появляются мнимые числа.

    \smallskip
    \noindent 2. Построить отрицание к данному высказыванию. \bigskip

    Решение:

    $$\lnot (\exists a : \forall b \  \exists x : x^2 + ax + b = 0) \Longrightarrow \forall
    a \ \exists b : \forall x \ x^2 + ax + b \neq 0$$

    
    \section{}
    Доказать, что $A \  \backslash \  (B \ \backslash \  C) = (A \ \backslash \ B) \cup (A \cap C)$

    Решение:
    \begin{align*}
        &x \in (A \ \backslash \ B) \cup (A \cap C) = x \in (A \ \backslash \ B) \ \vee \ x \in (A \cap C) =\\
        &=(x \in A \ \wedge \ x \notin B) \ \vee \ (x \in A \ \wedge x \in C) = x \in A \  \wedge \ (x \notin 
        B \ \vee \ x \in C) =\\
        &= x \in A \ \wedge \lnot  (x \notin B \ \vee \ x \in C) = x \in A \ \wedge x \notin (B \ \backslash \ C) =
        x \in A \ \backslash \ (B \ \backslash \ C)
    \end{align*}

    \section{}
    Высказывание $\forall x(\lnot P(x) \rightarrow (P(x) \ \vee \ \lnot(\lnot Q(x) \rightarrow P(x))))$ ложно. Докажите, что 
    $\forall x, P(x)$ ложно, а $\exists x : Q(x)$ истинно.

    Решение:


    
\end{document}