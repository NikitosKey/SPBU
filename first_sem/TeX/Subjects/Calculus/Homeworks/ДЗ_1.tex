\documentclass[a4paper, fontsize=10pt]{article}
% преамбула документа
\usepackage{scrextend}
\usepackage[left=2.0cm, right=1.5cm, top=2.0cm,
bottom=2.0cm]{geometry}
\usepackage[utf8]{inputenc}
\usepackage[T2A]{fontenc}
\usepackage{hyperref}
\usepackage[english, russian]{babel}
\usepackage{amsmath,amsfonts,amssymb}
\usepackage{graphicx}
\usepackage{setspace}
\usepackage{fontsize}
\usepackage{titlesec}
\titlelabel{\thetitle.\quad}
\usepackage{pgfplots}
\pgfplotsset{compat=1.15}
\usepackage{mathrsfs}
\usepackage{mathtools}
\usetikzlibrary{arrows}

\begin{titlepage}
    \title{Homework 1}
    \date{}
\end{titlepage}


\begin{document}
    \maketitle

    \section{\normalsize \normalfont Упростить формулу:}
    $$((P \longrightarrow Q) \longrightarrow P) \longrightarrow P$$

    Решение:

    \begin{align*}
        &((P \longrightarrow Q) \longrightarrow P) \longrightarrow P = \lnot(\lnot(\lnot P \ \vee \ Q) 
        \ \vee \ P) \ \vee \ P = ((\lnot P \ \vee \ Q) \ \wedge \ P) \ \vee \ P =(P \ \wedge \ Q) \ \vee \ P  
    \end{align*}


    \section{}

    1) Определить значение истинности высказывания, считая, что все переменные
    пробегают множество действительных чисел.

    $$\exists a : \forall b \  \exists x : x^2 + ax + b = 0$$ \smallskip

    Решение:
    %$x^2 + 2345678654567x + 10000000000000000000000000000000000000000 = 0  $
    $$x = - \frac{a}{2} - \frac{\sqrt{a^{2} - 4 b}}{2}, \  x = - \frac{a}{2} + \frac{\sqrt{a^{2} - 4 b}}{2}$$\smallskip
    При $a^2 < 4b$ данное выражение ложно, так как появляются мнимые числа.

    \smallskip
    \noindent 2) Построить отрицание к данному высказыванию. \bigskip

    Решение:

    $$\lnot (\exists a : \forall b \  \exists x : x^2 + ax + b = 0) \Longrightarrow \forall
    a \ \exists b : \forall x \ x^2 + ax + b \neq 0$$

    
    \section{\normalsize \normalfont Доказать, что:}
    $$A \  \backslash \  (B \ \backslash \  C) = (A \ \backslash \ B) \cup (A \cap C)$$

    Решение:
    \begin{align*}
        &x \in (A \ \backslash \ B) \cup (A \cap C) = x \in (A \ \backslash \ B) \ \vee \ x \in (A \cap C) =
        (x \in A \ \wedge \ x \notin B) \ \vee \ (x \in A \ \wedge x \in C) =\\ &= x \in A \  \wedge \ (x \notin B \
        \vee \ x \in C) = x \in A \ \wedge \lnot  (x \notin B \ \vee \ x \in C) = x \in A \ \wedge x \notin
        (B \ \backslash \ C) = \\ &=x \in A \ \backslash \ (B \ \backslash \ C)
    \end{align*}

    \section{\normalsize \normalfont Высказывание $\forall x(\lnot P(x) \rightarrow (P(x) \ \vee \ \lnot(\lnot Q(x) \rightarrow P(x))))$ ложно. Докажите, что 
    $\forall x, P(x)$ ложно, а $\exists x : Q(x)$ истинно.}


    Решение:

    \begin{align*}
        &\lnot P(x) \rightarrow (P(x) \ \vee \ \lnot(\lnot Q(x) \rightarrow P(x))) = 0 \Longrightarrow \overbrace{\lnot \underbrace
        {P(x)}_0}^1 \rightarrow \overbrace{(P(x) \ \vee \ \lnot(\lnot Q(x) \rightarrow P(x)))}^0 = 0 \Longrightarrow\\
        &\Longrightarrow (0 \ \vee \ \overbrace{\lnot(\lnot \underbrace{Q(x)}_1 \rightarrow 0))}^0 = 0 \Longrightarrow P(x) = 0, \ Q(x) = 1
    \end{align*}

    \section{\normalsize \normalfont Сколько подмножеств в множестве, состоящем из $n$ элементов?}

    Решение: 

    \begin{enumerate}
        \item Сколько различных подмножеств длины 1? 
        
        Если взять множество длинны $n$, то мы имеем только $n$ отличных подномжеств из одного элемента.

        Допустим, что это размещения $C_n^1 = \frac{n!}{1!(n-1)!} = n$

        \item Сколько для 2?
        
       $$C_n^2 = \frac{n!}{2!(n-2)!} = \frac{n (n - 1)}{2}$$

        \item Для $n-1$
        
        $$C_n^{n-1} = \frac{n!}{(n-1)!(n-n+1)!} = n$$

        \item Для $n$ - одно.
        
        $$C^k_n = \frac{n!}{n!(n-n)!} = 1$$
        
    
        \item  Количество множеств в подномжестве:
        
        $$ \sum_{k = 0}^{n} C_n^k$$
        
    \end{enumerate}



 


        
\end{document}