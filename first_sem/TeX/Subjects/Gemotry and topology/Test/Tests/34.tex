\documentclass[a4paper, fontsize=14pt]{article} % преамбула документа
\usepackage{scrextend}
\usepackage[left=2.0cm, right=1.5cm, top=2.0cm,
bottom=2.0cm]{geometry}
\usepackage[utf8]{inputenc}
\usepackage[T2A]{fontenc}
\usepackage{hyperref}
\usepackage[english, russian]{babel}
\usepackage{amsmath,amsfonts,amssymb}
\usepackage{graphicx}
\usepackage{setspace}
\usepackage{fontsize}
\usepackage{pgfplots}
\usepackage{pgfplots}
\pgfplotsset{compat=1.15}
\usepackage{mathrsfs}
\usetikzlibrary{arrows}


\begin{titlepage}
    \title{Контрольная по КВП}
    \author{Морозов Никита 23.Б09}
    \date{}
\end{titlepage}


\begin{document}

    \definecolor{ffqqqq}{rgb}{1.,0.,0.}
    
    \maketitle

    \centerline{Вариант 34} \bigskip

    \noindent Привести уравнение кривой второго порядка к каноническому виду, найти координаты центра и фокусов в 
    исходной системе координат и построить эскиз графика: $-3x^2 - 23y^2 - 48xy + 42x - 2y = 485$ . \bigskip

    Решение: \bigskip

    \begin{enumerate}
        \item Приведем к каноническому виду: $a_{11}x^2 + 2a_{12}xy + a_{22}y^2 + 2a_{13}x + 2a_{23}y + a_{33} = 0$
        
        \begin{equation} \label{Canoic}-3x^2 - 48xy - 23y^2 + 42x - 2y - 485 = 0 \end{equation}

        \item  Найдём центр кривой составив систему:
        
        $ 
        \begin{cases}
            a_{11}x_0 + a_{12}y_0 + a_{13} = 0\\
            a_{12}x_0 + a_{22}y_0 + a_{23} = 0
        \end{cases}
        $
        где $(x_0; y_0)$ - центр кривой

        Подставим коэфициенты из \eqref{Canoic}.
        \begin{equation} \label{Coordinates_system}
            \begin{cases}
                -3x_0 - 24y_0 + 21 = 0 \\
                -24x_0 - 23y_0 - 1 = 0
            \end{cases}
        \end{equation}

            $(-1;1)$ - центр кривой

        \item Перенесём начало координат в центр прямой.
        
        Подставим в \eqref{Canoic}:

        $$
        \begin{cases}
            x = x' -1\\
            y = y' + 1
        \end{cases}
        $$

        Получим:

        $$-3(x -1)^2 - 48(x - 1)(y + 1) - 23(y + 1)^2 + 42(x -1) - 2(y + 1)- 485 = 0$$

        $$ - 3 x^{2} - 48 x y - 23 y^{2} - 507 = 0$$
        
        \item 

       $$ \delta = \begin{vmatrix} -3 && -24\\ -24 && -23\\\end{vmatrix} = -3 \cdot (-23) - (-24 \cdot (-24)) = -507$$ 

       \begin{align*}
       & \Delta = 
       \begin{vmatrix}
            -3 & -24 & 21\\
            -24 & -23 & - 1\\
            \quad 21 & -1 & -485\\
       \end{vmatrix}
       = -3 ((-23 \cdot (- 485)) - (-1  \cdot (-1))) + 24 ((-24 \cdot (-485)) -\\
        & - (21 \cdot (-1))) + 21 ((-24 \cdot (-1)) - (21 \cdot (-23))) = 257049
       \end{align*}

       $\frac{\Delta}{\delta} = -507 \Longrightarrow$ Есть центр симетрии.

       $\delta < 0 \ \vee \ \Delta \neq 0 \Longrightarrow$ это гипербола

       \item Подберем коэфициенты в новой системе координат методом инвариантов.
       

    $$
    \begin{cases}
        I_1 = -3 -23 = -26 = a'_{11} + a'_{22}\\
        I_2 = \delta = -507 = a'_{11} \cdot a'_{22}\\
        I_3 = \Delta = 257049 = a'_{11} \cdot a'_{22} \cdot a'_{33}
    \end{cases} 
    $$

    $$
    [[a_{11} = -39, \  a_{22} = 13, \ a_{33} = -507], \ [a_{11} = 13, \  a_{22} = -39, \ a_{33} = -507]]
    $$

    Получим два варианта решения:

    1) $- \frac{x^{2}}{13} + \frac{y^{2}}{39} = 1$
    
    2) $\frac{x^{2}}{39} - \frac{y^{2}}{13} = 1$ \bigskip

    Их фокусы:

    1) $c = \sqrt{13 + 39} = 2 \sqrt{13} \Longrightarrow [(0, 2 \sqrt{13}), (0, -2 \sqrt{13})]$

    2) $c = \sqrt{13 + 39} = 2 \sqrt{13} \Longrightarrow [(2 \sqrt{13}, 0), (-2 \sqrt{13}, 0)] $
   
    \begin{figure}[b]
        \centering
        \begin{tikzpicture}[line cap=round,line join=round,>=triangle 45,x=0.5cm,y=0.5cm]
            \begin{axis}[
            x=0.5cm,y=0.5cm,
            axis lines=middle,
            ymajorgrids=true,
            xmajorgrids=true,
            xmin=-10.0,
            xmax=10.0,
            ymin=-10.0,
            ymax=10.0,
            xtick={-10.0,-8.0,...,10.0},
            ytick={-10.0,-8.0,...,10.0},]
            \clip(-10.,-10.) rectangle (10.,10.);
            \draw [samples=50,domain=-0.99:0.99,rotate around={90.:(0.,0.)},xshift=0.cm,yshift=0.cm,line width=2.pt,color=ffqqqq] plot ({6.244997998398398*(1+(\x)^2)/(1-(\x)^2)},{3.605551275463989*2*(\x)/(1-(\x)^2)});
            \draw [samples=50,domain=-0.99:0.99,rotate around={90.:(0.,0.)},xshift=0.cm,yshift=0.cm,line width=2.pt,color=ffqqqq] plot ({6.244997998398398*(-1-(\x)^2)/(1-(\x)^2)},{3.605551275463989*(-2)*(\x)/(1-(\x)^2)});
            \begin{scriptsize}
            \draw[color=ffqqqq] (-4.9181404179732375,12.822631462561157) node {$eq1$};
            \end{scriptsize}
            \end{axis}
            \end{tikzpicture}
            \label{first_plot}
            \caption{Первое решение}
    \end{figure}

    \begin{figure}[ht]
        \centering
        \begin{tikzpicture}[line cap=round,line join=round,>=triangle 45,x=0.5cm,y=0.5cm]
            \begin{axis}[
            x=0.5cm,y=0.5cm,
            axis lines=middle,
            ymajorgrids=true,
            xmajorgrids=true,
            xmin=-10.0,
            xmax=10.0,
            ymin=-10.0,
            ymax=10.0,
            xtick={-10.0,-8.0,...,10.0},
            ytick={-10.0,-8.0,...,10.0},]
            \clip(-10.,-10.) rectangle (10.,10.);
            \draw [samples=50,domain=-0.99:0.99,rotate around={0.:(0.,0.)},xshift=0.cm,yshift=0.cm,line width=2.pt,color=ffqqqq] plot ({6.244997998398398*(1+(\x)^2)/(1-(\x)^2)},{3.605551275463989*2*(\x)/(1-(\x)^2)});
            \draw [samples=50,domain=-0.99:0.99,rotate around={0.:(0.,0.)},xshift=0.cm,yshift=0.cm,line width=2.pt,color=ffqqqq] plot ({6.244997998398398*(-1-(\x)^2)/(1-(\x)^2)},{3.605551275463989*(-2)*(\x)/(1-(\x)^2)});
            \begin{scriptsize}
            \draw[color=ffqqqq] (12.848903952263086,5.647478928427263) node {$eq1$};
            \end{scriptsize}
            \end{axis}
            \end{tikzpicture}
            \label{second_plot}
            \caption{Второе решение}
    \end{figure}





    \end{enumerate}


\end{document}