\documentclass[a4paper, fontsize=10bp]{article} % преамбула документа
\usepackage{scrextend}
\usepackage[left=2.0cm, right=1.5cm, top=2.0cm,
bottom=2.0cm]{geometry}
\usepackage[utf8]{inputenc}
\usepackage[T2A]{fontenc}
\usepackage{hyperref}
\usepackage[english, russian]{babel}
\usepackage{amsmath,amsfonts,amssymb}
\usepackage{graphicx}
\usepackage{setspace}
\usepackage{fontsize}

\begin{document}

\section*{Указ Президента Российской Федерации от 07.05.2018 г. № 204}

\begin{itemize}
    \item В 2024 году Россия должна войти в число пяти ведущих стран мира, осуществляющих научные исследования и разработки.
    \item Необходимо обеспечить присутствие Российской Федерации в числе пяти ведущих стран мира, осуществляющих научные исследования и разработки в областях, определяемых приоритетами научно-технологического развития.
    \item Обеспечение привлекательности работы в Российской Федерации для российских и зарубежных ведущих ученых и молодых перспективных исследователей.
    \item Опережающее увеличение внутренних затрат на научные исследования и разработки за счет всех источников по сравнению с ростом валового внутреннего продукта.
    \item Создание передовой инфраструктуры научных исследований, включая уникальную установку класса мегасайенс.
    \item Создание эффективной системы высшего образования, обеспечивающей подготовку кадров для современной экономики.
    \item Создание условий для воспитания гармонично развитой и социально ответственной личности на основе духовно-нравственных ценностей народов Российской Федерации, исторических и национально-культурных традиций.
    \item Увеличение численности населения Российской Федерации, повышение ожидаемой продолжительности жизни до 78 лет (к 2030 году - до 80 лет).
    \item Создание устойчивой системы обращения с твердыми коммунальными отходами, обеспечивающей сортировку отходов в объеме 100 процентов и снижение их захоронения в два раза.
\end{itemize}



\section*{ Указ Президента Российской Федерации от 07.05.2012 г. № 596 }


    \quad \ \ {\bfseries Цель:}\ повышение темпов экономического роста, увеличение реальных доходов граждан, технологическое лидерство.

    {\bfseries Задачи:}\ создание и модернизация 25 млн. высокопроизводительных рабочих мест, увеличение инвестиций до 25\% ВВП, увеличение доли высокотехнологичных отраслей, повышение производительности труда, повышение позиции России в рейтинге Всемирного банка.

    {\bfseries Правительству РФ:}\ утверждение основных направлений деятельности, подготовка федерального закона о государственном стратегическом планировании, утверждение основных государственных программ, упрощение бухгалтерской отчетности, обеспечение обязательного публичного обсуждения размещаемых заказов, повышение прозрачности финансовой деятельности, приватизация и совершенствование управления государственным имуществом, улучшение условий ведения предпринимательской деятельности, модернизация и инновационное развитие экономики.

\end{document}