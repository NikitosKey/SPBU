
%\setcounter{chapter}{-1}


\chapter{Комплексные числа}

\section{Алгебраическая форма}

\begin{definition}
    
    Коплексным числом $z$ называют сумму действительного числа $a$ и чисто мнимого числа $i \cdot b$

    $ z = a + i \cdot b $, где $ z \ni \mathbb{C}$, $a \ni \mathbb{R}$, $b \ni \mathbb{R}$

    $i^2 = -1$

    $a = Re z, b = Im z$
\end{definition}

\subsection*{Действия}

\begin{itemize}

    \item Сложение
    \begin{quotation}

        $z_1 + z_2 = (a_1 + i \cdot b_1) + (a_2 + i \cdot b_2)$

        $z_1 + z_2 = a_1 + a_2 + (b_1 + b_2) \cdot i$
    \end{quotation}
    

    \item Умножение
    \begin{quotation}
        $z_1 \cdot  z_2 = (a_{1} + i \cdot b_1)(a_2 + i \cdot b_2)$

    $z_1 \cdot  z_2  =(a_1 \cdot a_2 - b_1 \cdot b_2) + (b_1 \cdot a_2 + b_2 \cdot a_1) \cdot i$
    \end{quotation}
    


    \item Вычитание
    \begin{quotation}
        $ \frac{z_1}{z_2} = \frac{a_1 + i \cdot b_1}{a_2 + b_2 \cdot i} = \frac{a_1 \cdot a_2 + b_1 \cdot b_2}{a_2^2 + b_2^2} + \frac{b_1 \cdot a_2 - a_1 \cdot b_2}{a_2^2 + b_2^2} \cdot i$
    \end{quotation}

\end{itemize}

\section{Геометрическое представление}

\section{Тригонометрическая форма}

\begin{definition}

    
\end{definition}

\task{}






